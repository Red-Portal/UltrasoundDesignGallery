
\section{Introduction}\label{section:introduction}
\IEEEPARstart{M}{edical} ultrasound B-mode (brightness mode) images require various stages of image processing in order to be clinically efficient.
compared to other medical image modalities such as magnetic resonance imaging (MRI), without post-processing treatments, medical ultrasound images suffer from low signal-to-noise response and low contrast.
For this reason, various image enhancement methods including both purely image processing-based and system-wise optimization methods have been developed~\cite{contrerasortiz_ultrasound_2012}.

Purely image processing based methods such as \textit{speckle reduction}~\cite{finn_echocardiographic_2011, duarte-salazar_speckle_2020} has shown to be able to drastically improve the quality of medical ultrasound images without increasing the hardware complexity of medical ultrasound devices. 
Despite their advantages, image processing methods introduce many system parameters that strongly affect the finally formed image~\cite{duarte-salazar_speckle_2020}.
Appropriate tuning of these parameters is crucial for extracting their best clinical performance and proper performance assessment.

When no theoretically supported values for system parameters are available, a typical approach is to set a target quality metric and optimize it either manually or using numerical optimization methods. For instance, Ramos-Llord\'en et al.~\cite{ramos-llorden_anisotropic_2015} maximized the~\(\widehat{Q}\) index~\cite{tay_ultrasound_2006} while Mishra et al.~\cite{mishra_edge_2018} minimized the signal-to-speckle-noise ratio (SSRN).
Unfortunately, for medical ultrasound B-mode images, this approach is limited by several critical reasons.
\begin{enumerate}
  \item Commonly used objective quality metrics are not entirely aligned with human perception.
  \item Subjective quality assessments differ greatly across radiologists and sonographers.
  \item Furthermore, such quality assessments might change depending on the clinical objective.
\end{enumerate}
For example, Outtas et al.~found that ``\ldots the contrast perceived by the radiologists is far from the one assessed by the three objective metrics used'' in their experiments~\cite{outtas_subjective_2018}.
A perhaps extreme example of this discrepancy between objective quality metrics and actual clinical performance is in the task of speckle reduction.
In~\cite{loizou_comparative_2005}, Loizou et al.~note that speckle ``\ldots is not truly noise in the typical engineering sense because its texture often carries useful information about the image being viewed.''.
However, most objective quality metrics used in speckle reduction simply focus in \textit{reducing} speckle.

Such mismatch between objective quality metrics and actual clinical performance has resulted some works in ultrasound image enhancement to utilize subjective quality assessments~\cite{loizou_quality_2006, hemmsen_ultrasound_2010, wong_monte_2012, kang_new_2016, mishra_edge_2018}.
(For a review on subjective quality metrics in medical imaging not restricted to ultrasound, see~\cite{chow_review_2016}.)
While this practice of using subjective metrics enable clinically-calibrated comparison of individual image enhancement methods, they are still inappropriate to use for \textit{tuning} system parameters.

Another important issue is that even if we have access to reliable quality assessments, the large number of parameters in image enhancements methods complicates tuning.
(See Table 5 in~\cite{finn_echocardiographic_2011} for an incomplete list of such parameters.)
This number grows as we start combining multiple image enhancement methods.
It is therefore crucial to automate the navigation of these high-dimensional parameter spaces. 

To solve the problem of tuning the parameters of image enhancement methods, we propose the~\usdg, a graphical tool for tuning medical ultrasound image enhancement methods using subjective quality assessment.
The Ultrasound Design Gallery is based on the \textit{design gallery} interface~\cite{brochu_bayesian_2010, 10.1145/3072959.3073598, koyama_sequential_2020, phan_color_2018, pmlr-v119-mikkola20a} and Bayesian optimization (BO,~\cite{shahriari_taking_2016}), a gradient-free global optimization algorithm.

\cite{deng_speckle_2011, wong_monte_2012, hu_cluster_2016, singh_hybrid_2017, nagare_multi_2017} tend to generate blurry images regardless of the speckle reduction performance.
Especially, despite the superior performance of the hybrid filter~\cite{singh_hybrid_2017}.

reporting the exact image dimensions and view depths is important since they affect the sampling rate the denoising algorithms operate.
For example, the window size of the DPAD coefficient.
They tend to severly affect the performance of despeckling filters and make proper comparison difficult.

Methods such as~\cite{hutchison_probabilisticdriven_2010, bini_despeckling_2014} exploit local homogeneity.
However, these methods are difficult to implement in a real-time fashion on highly parallel computing hardware such as GPUs.

Speckle reduction algorithms result in improved contrast and higher lesion detectibility~\cite{bottenus_resolution_2021}.
The recently introduced generalized contrast-to-noise ratio metric~\cite{rodriguez-molares_generalized_2020}.

Laplace regularized PMAD

%and have been integrated into  types of human-computer interfaces~\cite{brochu_bayesian_2010, 10.1145/3072959.3073598, koyama_sequential_2020, phan_color_2018, pmlr-v119-mikkola20a}.

%% For example, in~\cite{ramos-llorden_anisotropic_2015}, Ramos-Llorden et al.~minimized the \(\widehat{Q}\) index for tuning the parameters of various speckle reduction algorithms.

%% While in general optimizing image quality metrics
%% tuning these parameters 
%% However, 
%% While optimization methods are to the rescue 
%% For this reason, 

%% ``\ldots the contrast perceived by the radiologists is far from the one assessed by the three objective metrics used''~\cite{outtas_subjective_2018}.

%% Tools for aiding visual designs have been employed in other fields such as computer graphics~\cite{10.1145/258734.258887}.

%% Despite showing strong speckle reuduction properties, these methods require excessive and tedious tuning~\cite{duarte-salazar_speckle_2020}.
%% For example,~\cite{ramos-llorden_anisotropic_2015} optimized the \(\widehat{Q}\) criterion (originally proposed in~\cite{tay_ultrasound_2006}) for tuning parameters.
%% However, ``OSRAD and POSRAD filters are not able to improve the DPAD filter result mainly due to the higher preservation of structures in the background class'' and ``\ldots in the following experiments with real images, \ldots over-filtering in background images does not necessarily lead to favorable visual results.''.

%% In this paper, we present 

%% \cite{hemmsen_ultrasound_2010} pairwise and continuous comparison however does not perform automatic tuning based on the user feedback.

%% Human-computer interaction for visual design~\cite{tory_human_2004}

%% For high-dimensional spaces,~\cite{10.1145/3386569.3392409} use random linear embeddings~\cite{10.5555/2540128.2540383, NEURIPS2020_10fb6cfa}.


The advantages of \usdg~are summarized as follows:
\begin{itemize}
  \item It uses a slider based graphical interface that enables efficient communication between the \user~and our system (\textbf{\cref{section:ui}}).
  \item It learns the \user's image quality preferences using Gaussian processes (\textbf{\cref{section:gp}}).
  \item It uses preferential Bayesian optimization for efficiently navigating high-dimensional, non-linear parameter spaces of medical ultrasound image pipelines (\textbf{\cref{section:bo}}).
  \item It enables effecient task, \user~specific tuning of medical ultrasound image processing pipelines.
\end{itemize}

In addition, this paper has the following contributions:
\begin{itemize}
  \item We discuss efficient implementation techiques of projective preferential Bayesian optimization.
  \item We apply the \usdg for tuning an ultrasound image enhancement algorithm based on multiscale nonlinear anisotropic diffusion.

\end{itemize}

%%% Local Variables:
%%% TeX-master: "master"
%%% End:
