
\section{Dissusions}\label{section:conclusion}
Understanding and optimizing the subjective quality metrics of sonographers is a crucial task for improving the clinical performance of our medical ultrasound imaging systems.
In this work, we have presented the \textsc{Ultrasound Design Gallery}, a graphical tool for inferring and optimizing the subjective metrics of sonographers.
By leveraging probabilistic machine learning and Bayesian optimization, our tool enables automatic, personalized, task-specific tuning of medical ultrasound imaging systems.
We have demonstrated the utility of our tool by letting sonographers tune the parameters of the \textsc{cascaded laplacian pyramid diffusion} (CLPD), which is a novel medical ultrasound image enhancement algorithm.
The CLPD enables sharing of information between different image scales and seemless integration of conventional image enhancement filters.

Our experimental results have shown that individual sonographers exhibit different visual preferences.
Also, their preference vary across clinical tasks and scanning views.
Therefore, personalized and task-specific tuning of ultrasound image enhancement algorithms, possibly using the USDG, is crucial for optimal clinical performance.

While image sharpness/blurriness have not been a focus in medical ultrasound images, we have shown that sonographers prioritize natural and sharp looking images.
Unfortunately, most conventional blurriness metrics are calibrated to natural images.
Therefore, we point out that quantifying the blurriness of medical ultrasound images is an intersting open problem.
%In addition, for understanding and optimizing the preference of sonographers and radiologists, methods developed in psychology such as~\cite{NIPS2007_89d4402d} could be interesting to explore.
%Other advanced Bayesian optimization methods would also be worth exploring.

%%% Local Variables:
%%% TeX-master: "master"
%%% End:
