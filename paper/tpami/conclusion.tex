
\section{Dissusion}\label{section:conclusion}
The performance metrics used in medical ultrasound imaging have been shown to be not calibrated with the subjective metrics of sonographers.
Therefore, understanding and optimizing the subjective metrics of sonographers is a crucial task for improving the clinical performance of our medical ultrasound imaging systems.
In this work, we have presented the \textsc{Ultrasound Design Gallery}, a graphical tool for inferring and optimizing the subjective preference metrics of sonographers.
By leveraging probabilistic machine learning and Bayesian optimization, our tool enables automatic, personalized, task-specific tuning of medical ultrasound imaging systems.
We have demonstrated the utility of our tool by letting sonographers tune the parameters of the cascaded laplacian pyramid diffusion filter.
We have also contributed a new image-enhancement algorithm, the cascaded laplacian pyramid diffusion.
Based on the classica Laplacian pyramid, it enables sharing information between different scales and seemless integration of conventional image enhancement filters.

Our experimental results have shown that individual sonographers exhibit different visual preferences.
Also, the same sonographers show different preference depending on the clinical task and scanning view.
Therefore, personalized, task-specific tuning of image enhancement algorithms, possibly using the Ultrasound Design Gallery, is crucial for optimal clinical performance.
Moreover, we have shown that sonographers show strong preference towards natural and sharp looking images.
Until now, image sharpness/blurriness have not been a focus in medical ultrasound images.
Unfortunately, most conventional blurriness metrics are calibrated on natural images.
Thus, appropriately quantifying the blurriness of medical ultrasound images is an open problem that will need investigating in the future.
Last but not least, we emphasize the importance of understanding and optimizing the preference of sonographers and radiologist.
Methods developed in psychology such as~\cite{NIPS2007_89d4402d} could be use to infer subjective preference, while other advanced Bayesian optimization methods would be worth exploring.

%%% Local Variables:
%%% TeX-master: "master"
%%% End:
