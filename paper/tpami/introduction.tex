
\section{Introduction}\label{section:introduction}
\IEEEPARstart{M}{edical} ultrasound B-mode (brightness mode) images require various stages of image processing in order to be clinically effective.
When compared against other medical image modalities such as magnetic resonance imaging (MRI), without proper post-processing treatments, medical ultrasound images suffer from low signal-to-noise response and low contrast.
For this reason, various image processing-based and system-wise image enhancement methods have been pioneered~\cite{contrerasortiz_ultrasound_2012}.

Purely image processing-based methods such as \textit{speckle reduction}~\cite{finn_echocardiographic_2011, duarte-salazar_speckle_2020} has shown to drastically improve the quality of medical ultrasound images without increasing the hardware complexity of medical ultrasound devices. 
However, despite their advantages, image processing methods introduce many system parameters that strongly affect the resulting image~\cite{duarte-salazar_speckle_2020}.
Appropriate tuning of these parameters is crucial for extracting their best clinical performance.% and proper performance assessment.

Futhermore, in practice, ultrasound image tuning is a significant task both in the industry and clinics.
In clinics, newly commissioned ultrasound devices have to go through a significant amount of tuning before actually being deployed.
Despite such effort, since different sonographers and radiologists perceive images differently, any fixed parameter setting is poised to be suboptimal.
In the industry, proper tuning of image enhancement pipelines is crucial for the final performance of the overall system.
Therefore, companies invest a significant effort for parameter tuning both in terms of time and human resource.

When no analytical guideline for setting a certain parameter is available, a typical procedure is to set a target quality metric and optimize it either manually or automatically using numerical optimization.
For instance, Ramos-Llord\'en et al.~\cite{ramos-llorden_anisotropic_2015} maximized the~\(\widehat{Q}\) index~\cite{tay_ultrasound_2006} while Mishra et al.~\cite{mishra_edge_2018} minimized the signal-to-speckle-noise ratio (SSRN).
Unfortunately, for medical ultrasound B-mode images, this approach is limited by several critical reasons.
\vspace{0.05in}
\begin{enumerate}
  \item[\ding{228}] Commonly used objective quality metrics are not aligned with human perception.
    \vspace{0.05in}
  \item[\ding{228}] Subjective quality assessments differ greatly across doctors and sonographers.
    \vspace{0.05in}
  \item[\ding{228}] Furthermore, such subjective quality metric are likely to be task-dependent and change depending on the clinical objective.
\end{enumerate}
For example, Outtas et al.~found that ``\ldots the contrast perceived by the radiologists is far from the one assessed by the three objective metrics used'' in their experiments~\cite{outtas_subjective_2018}.
A perhaps extreme example of this discrepancy between objective quality metrics and actual clinical performance is in the task of speckle reduction.
In~\cite{loizou_comparative_2005}, Loizou et al.~note that speckle ``\ldots is not truly noise in the typical engineering sense because its texture often carries useful information about the image being viewed.''.
However, most objective quality metrics used in speckle reduction simply focus on entirely \textit{removing} speckle.

Such mismatch between objective quality metrics and actual clinical performance has resulted some works in ultrasound image enhancement to utilize subjective quality assessments~\cite{loizou_quality_2006, hemmsen_ultrasound_2010, wong_monte_2012, kang_new_2016, mishra_edge_2018}.
(For a review on subjective quality metrics in medical imaging not restricted to ultrasound, see~\cite{chow_review_2016}.)
While the practice of using subjective metrics enable clinically-calibrated comparison of individual image enhancement methods, they are still inappropriate to use for \textit{tuning} system parameters.
Also, even if we have access to reliable quality assessments, the large number of parameters in image enhancements methods complicates tuning.
(See Table 5 in~\cite{finn_echocardiographic_2011} for an incomplete list of such parameters.)
This number grows as we start combining multiple image enhancement methods.
%Therefore, a crucial problem in medical ultrasound image enhancement is to automate the navigation of these high-dimensional parameter spaces. 

In this paper, we propose the~\textsc{Ultrasound Design Gallery}, a machine learning based tool for automating medical image parameter tuning.
%The Ultrasound Design Gallery learns and optimizes the subjective quality metrics of clinical practitioners only using a simple graphical user interface.
%(in the context of ultrasound, sonographers).
\begin{itemize}
    \item[\ding{228}] First, the~\usdg~receives preference feedback from a sonographer through a \textit{design gallery}~\cite{brochu_bayesian_2010, 10.1145/3072959.3073598, koyama_sequential_2020, phan_color_2018, pmlr-v119-mikkola20a} user interface (\textbf{\cref{section:ui}}).
    \vspace{0.02in}
  \item[\ding{228}] Next, it learns the subjective quality metric of the sonographer from the preference feedback using a latent Gaussian process (GP,~\cite{rasmussen_gaussian_2006, pmlr-v119-mikkola20a}) probabilistic machine learning model (\textbf{\cref{section:gp}}).
    \vspace{0.02in}
  \item[\ding{228}] The subjective quality metric is optimized using Bayesian optimization (BO,~\cite{shahriari_taking_2016}), a black-box global optimization algorithm (\textbf{\cref{section:bo}}).
\end{itemize}
The~\usdg~enables automatic tuning of image enhancement algorithms personalized to the individual sonographers.
The sonographer does not need to have \textit{any knowledge about the image enhancement algorithm and its parameters}.

%% Overall, the advantages of the \usdg~are summarized as follows:
%% \begin{itemize}
%%   \item[\ding{228}] It provides and intuitive interface for communicating a sonographer's preference to the internal algorithm.
%%     \vspace{0.02in}
%%   \item[\ding{228}] It enables automatic navigation of the high-dimensional, non-linear parameter spaces of image enhancement algorithms.
%%     \vspace{0.02in}
%%   \item[\ding{228}] It enables effecient task, sonographer specific tuning of medical ultrasound image enhancement algorithms.
%% \end{itemize}

To evaluate the practicality of the \usdg, we use it for tuning the parameters of a novel ultrasound image enhancement algorithm, \textsc{cascaded Laplacian pyramid diffusion}.
Our proposed method is based on Laplacian pyramids~\cite{burt_laplacian_1983} and anisotropic diffusion~\cite{perona_scalespace_1990, weickert_anisotropic_1998}.
Previous methods~\cite{zhang_multiscale_2006, zhang_nonlinear_2007, kang_new_2016} that combine the two directly applied diffusion filters to the Laplacian band-pass images in \textit{parallel form}.
However, diffusion filters were not designed to be applied to band-pass images and this fact has left a fundamental ackwardness.
In contrast, we propose to apply the diffusion filters in \textit{cascaded form}.
Our novel scheme enables the diffusion filters to be applied to the band-limited images instead of band-pass images.
The cascaded Laplacian pyramid diffusion enables both structural enhancements and noise reduction without over-smoothing~\cite{ramos-llorden_anisotropic_2015, mishra_edge_2018} the ultrasound images.

We recruited five practicing sonographers and a cardiologist (whom will be also refered as a sonographer for conciseness) for tuning the cascaded Laplacian pyramid diffusion using the~\usdg~on abdominal and echocardiographic ultrsaound images.
We analyze the resulting parameters and study the visual preference of sonographers, where results suggest that commonly used objective performance metrics for ultrasound images (such as the SSNR) are not aligned with the preference of sonographers.
In the same time, according to \(S_3\)~\cite{vu_bf_2012}, a metric for assessing the blurriness of natural images, all of the involved sonographers strongly preferred less blurry, sharp looking images.
This suggests that image blurriness metrics should be included for evaluating ultrasound image enhancement algorithms.

\noindent Overall, we provide the following technical contributions:
\begin{itemize}
  \item[\ding{228}] \textsc{\textbf{Ultrasound Design Gallery}}: We present a graphical tool that can automatically and personally tune medical ultrasound image enhancement algorithms using the direct feedback of sonographers (\textbf{\cref{section:usdg}}).
    \vspace{0.02in}
  \item[\ding{228}] \textsc{\textbf{Cascaded Laplacian Pyramid Diffusion}}:  We propose a novel ultrasound image enhancement algorithm that incorporates anisotropic diffusion algorithms into Laplacian pyramid in cascaded form (\textbf{\cref{section:filter}}).
  \item[\ding{228}] We analyze the visual preference of sonographers by letting them tune the cascaded Laplacian pyramid diffusion using the Ultrasound Design Gallery (\textbf{\cref{section:eval}}).
\end{itemize}
All the code used in our work including the Ultrasound Design Gallery and the cascaded Laplacian pyramid diffusion (CUDA and Julia implementations) are openly available online\footnote{open source github repository: \url{https://github.com/Red-Portal/UltrasoundDesignGallery}}.

%% The interface of the \usdg~is primarily inspired by 
%% Among many different types of design galleries, \usdg~utilizes the \textit{sequential line search} interface designed by~\cite{10.1145/3072959.3073598}.

%% Design galleries were first developed in the computer graphics community of efficient visual design.

%% While visual tools for understanding the preference of sonographers have been introduced before~\cite{hemmsen_ultrasound_2010}, the \usdg~differs in that it provides 
%% focuses on optimizing and desiging image enhancement algorithms.
%% and  (BO,), a gradient-free global optimization algorithm.

%% reporting the exact image dimensions and view depths is important since they affect the sampling rate the denoising algorithms operate.
%% For example, the window size of the DPAD coefficient.
%% They tend to severly affect the performance of despeckling filters and make proper comparison difficult.

%% Methods such as~\cite{hutchison_probabilisticdriven_2010, bini_despeckling_2014} exploit local homogeneity.
%% However, these methods are difficult to implement in a real-time fashion on highly parallel computing hardware such as GPUs.

%% Speckle reduction algorithms result in improved contrast and higher lesion detectibility~\cite{bottenus_resolution_2021}.
%% The recently introduced generalized contrast-to-noise ratio metric~\cite{rodriguez-molares_generalized_2020}.

%% Laplace regularized PMAD

%and have been integrated into  types of human-computer interfaces~\cite{brochu_bayesian_2010, 10.1145/3072959.3073598, koyama_sequential_2020, phan_color_2018, pmlr-v119-mikkola20a}.

%% For example, in~\cite{ramos-llorden_anisotropic_2015}, Ramos-Llorden et al.~minimized the \(\widehat{Q}\) index for tuning the parameters of various speckle reduction algorithms.

%% While in general optimizing image quality metrics
%% tuning these parameters 
%% However, 
%% While optimization methods are to the rescue 
%% For this reason, 

%% ``\ldots the contrast perceived by the radiologists is far from the one assessed by the three objective metrics used''~\cite{outtas_subjective_2018}.

%% Tools for aiding visual designs have been employed in other fields such as computer graphics~\cite{10.1145/258734.258887}.

%% Despite showing strong speckle reuduction properties, these methods require excessive and tedious tuning~\cite{duarte-salazar_speckle_2020}.
%% For example,~\cite{ramos-llorden_anisotropic_2015} optimized the \(\widehat{Q}\) criterion (originally proposed in~\cite{tay_ultrasound_2006}) for tuning parameters.
%% However, ``OSRAD and POSRAD filters are not able to improve the DPAD filter result mainly due to the higher preservation of structures in the background class'' and ``\ldots in the following experiments with real images, \ldots over-filtering in background images does not necessarily lead to favorable visual results.''.

%% In this paper, we present 

%% \cite{hemmsen_ultrasound_2010} pairwise and continuous comparison however does not perform automatic tuning based on the user feedback.

%% Human-computer interaction for visual design~\cite{tory_human_2004}

%% For high-dimensional spaces,~\cite{10.1145/3386569.3392409} use random linear embeddings~\cite{10.5555/2540128.2540383, NEURIPS2020_10fb6cfa}.


%%% Local Variables:
%%% TeX-master: "master"
%%% End:
