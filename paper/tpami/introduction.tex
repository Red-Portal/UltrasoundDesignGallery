
\section{Introduction}\label{section:introduction}
\IEEEPARstart{M}{edical} ultrasound B-mode (brightness mode) images require various stages of image processing in order to be clinically effective.
Without proper image postprocessing treatment, ultrasound images suffer from low signal-to-noise response and low contrast.
Therefore, various image enhancement techniques have been developed~\cite{contrerasortiz_ultrasound_2012}.
%
Classical image enhancement approaches such as \textit{speckle reduction}~\cite{finn_echocardiographic_2011, duarte-salazar_speckle_2020} have shown to drastically improve the quality of ultrasound images without significantly increasing the hardware complexity of ultrasound devices. 
However, these often require sophisticated management of many parameters that strongly affect the resulting image~\cite{duarte-salazar_speckle_2020}.
Appropriately tuning these parameters is a crucial step for obtaining the best clinical performance.% and proper performance assessment.

While we will focus on the parameters of image enhancement algorithms in this work, tuning ultrasound imaging systems is an important, but nontrivial task both in industry and clinics.
In clinics, newly commissioned ultrasound devices have to go through a significant amount of tuning before being deployed.
Despite such effort, since different sonographers and radiologists perceive images differently, any fixed parameter setting is doomed to be unsatisfactory to some users.
In industry, the system parameters of an ultrasound imaging product crucially affects its final performance.
For high-end ultrasound scanners, companies invest 20\% to 40\% of the total development cost on parameter tuning\footnote{Private communication with two undisclosed medical ultrasound device manufacturers}.

A typical procedure for parameter tuning is to select a target quality metric and optimize it manually or automatically using numerical optimization.
For instance, Ramos-Llord\'en \textit{et al.}~\cite{ramos-llorden_anisotropic_2015} maximized the~\(\widehat{Q}\) index~\cite{tay_ultrasound_2006} while Mishra \textit{et al.}~\cite{mishra_edge_2018} maximized the signal-to-speckle noise ratio (SSNR).
Unfortunately, for medical ultrasound B-mode images, this seemingly natural approach is limited by the following reasons.
\vspace{0.05in}
\begin{enumerate}
  \item[\ding{228}] Objective quality metrics such as \(\widehat{Q}\) and the SSNR often fail to correlate with human perception.
    \vspace{0.05in}
  \item[\ding{228}] Subjective quality assessments differ greatly across sonographers and radiologists.
    \vspace{0.05in}
  \item[\ding{228}] Quality metrics need to adapt to the clinical objective and environment, which is difficult with objective metrics.
\end{enumerate}
For example, Outtas \textit{et al.}~\cite{outtas_subjective_2018} found that ``\ldots the contrast perceived by radiologists is far from the one assessed by the three objective metrics used'' in their experiments.
Perhaps an extreme example of this discrepancy between objective quality metrics and clinical performance is in the task of speckle reduction.
Loizou \textit{et al.}~\cite{loizou_comparative_2005} noted that speckle ``\ldots is not truly noise in the typical engineering sense because its texture often carries useful information about the image.''.
However, speckle reduction assessed under objective quality metrics generally pursues to \textit{entirely remove} speckle patterns.

To resolve the discrepancy between objective quality metrics and actual clinical performance, some works in ultrasound image enhancement have reported \textit{subjective} quality assessment results~\cite{loizou_quality_2006, hemmsen_ultrasound_2010, wong_monte_2012, kang_new_2016, mishra_edge_2018}.
(For a review on subjective quality metrics in medical imaging, not restricted to ultrasound, see~\cite{chow_review_2016}.)
While subjective metrics enable clinically-calibrated comparison of individual image enhancement methods, they are still difficult to use for tuning.
Previous approaches have attempted to \textit{learn} the subjective metrics~\cite{el-zehiry_learning_2013, abdi_automatic_2017, annangi_ai_2020}, requiring a large dataset of quality assessments.
That is, tuning a new system on a new task involves gathering a whole new dataset from scratch.
Also, even if we have access to a dataset of quality assessments, the large number of parameters in ultrasound image enhancement algorithms complicates tuning.
(See~\cite[Table 5]{finn_echocardiographic_2011} for an incomplete list of examples.)
%This number multiplies as we combine multiple image processing algorithms.
%Therefore, a crucial problem in medical ultrasound image enhancement is to automate the navigation of these high-dimensional parameter spaces. 

In this paper, we propose the~\textsc{Ultrasound Design Gallery} (USDG), a machine learning-based tool for tuning ultrasound image enhancement algorithms.
%The Ultrasound Design Gallery learns and optimizes the subjective quality metrics of clinical practitioners only using a simple graphical user interface.
%(in the context of ultrasound, sonographers).
\begin{itemize}
    \item[\ding{228}] The~\usdg~receives preference feedback from the sonographer through a \textit{design gallery}~\cite{brochu_bayesian_2010, 10.1145/3072959.3073598, koyama_sequential_2020, phan_color_2018, pmlr-v119-mikkola20a} interface (\textbf{\cref{section:ui}}).
    \vspace{0.02in}
  \item[\ding{228}] Next, it infers the subjective quality metric of the sonographer from the preferential feedback using a latent Gaussian process (GP,~\cite{rasmussen_gaussian_2006, pmlr-v119-mikkola20a}) probabilistic machine learning model (\textbf{\cref{section:gp}}).
    \vspace{0.02in}
  \item[\ding{228}] The inferred quality metric is optimized using Bayesian optimization (BO,~\cite{shahriari_taking_2016}), a black-box global optimization algorithm (\textbf{\cref{section:bo}}).
\end{itemize}
The~\usdg~enables personalization of image enhancement algorithms to individual sonographers and clinical tasks.
The sonographer does not need to have any knowledge about the image enhancement algorithm being tuned and its parameters while interacting with the USDG.
Unlike the autotuning approach of El-Zehiry \textit{et al.}~\cite{el-zehiry_learning_2013}, the USDG does not require a large dataset of subjective quality assessments.
Tuning a system from scratch takes a mere few tens of minutes.

%% Overall, the advantages of the \usdg~are summarized as follows:
%% \begin{itemize}
%%   \item[\ding{228}] It provides and intuitive interface for communicating a sonographer's preference to the internal algorithm.
%%     \vspace{0.02in}
%%   \item[\ding{228}] It enables automatic navigation of the high-dimensional, non-linear parameter spaces of image enhancement algorithms.
%%     \vspace{0.02in}
%%   \item[\ding{228}] It enables effecient task, sonographer specific tuning of medical ultrasound image enhancement algorithms.
%% \end{itemize}

To evaluate the effectiveness of the \usdg, we use it for tuning the parameters of a novel ultrasound image enhancement algorithm, the \textsc{cascaded Laplacian pyramid diffusion} (CLPD).
The CLPD is based on Laplacian pyramids~\cite{burt_laplacian_1983} and anisotropic diffusion~\cite{perona_scalespace_1990, weickert_anisotropic_1998}.
Previous methods~\cite{zhang_multiscale_2006, zhang_nonlinear_2007, kang_new_2016} combined the two methods by applying anisotropic diffusion to each Laplacian band-pass image in \textit{parallel form}.
However, anisotropic diffusion filters were not devised to be applied to band-pass images, raising concerns on compatibility.
%have not been developed to be 
%expected to be applied to 
%are not compatible with .
Also, the parallel form does not allow information to be shared between different image scales.
We instead propose to use Laplacian pyramids in \textit{cascaded form}.
This allows anisotropic diffusion filters to no longer be applied to band-pass images.
Also, feature information can naturally flow from higher scales to lower scales.
The cascaded Laplacian pyramid diffusion enables both structural enhancements and noise reduction without over-smoothing~\cite{ramos-llorden_anisotropic_2015, mishra_edge_2018} the ultrasound images.

We recruited five sonographers and a cardiologist for tuning the CLPD using the~\usdg~on abdominal and echocardiographic ultrasound images.
Our results afirm that commonly used objective performance metrics for ultrasound images (such as the SSNR) are not aligned with sonographers' preferences.
Instead, we show that all of the involved sonographers strongly preferred less blurry, sharp-looking images according to \(S_3\)~\cite{vu_bf_2012}, an image blurriness metric.

Overall, we provide the following technical contributions:
\begin{itemize}
  \item[\ding{228}] \textsc{\textbf{Ultrasound Design Gallery}}: We present a graphical tool that can automatically and personally tune medical ultrasound image enhancement algorithms using the direct feedback of sonographers (\textbf{\cref{section:usdg}}).
    \vspace{0.02in}
  \item[\ding{228}] \textsc{\textbf{Cascaded Laplacian Pyramid Diffusion}}:  We propose a novel ultrasound image enhancement algorithm that incorporates anisotropic diffusion algorithms into Laplacian pyramid in cascaded form (\textbf{\cref{section:filter}}).
  \item[\ding{228}] We analyze the visual preference of practicing sonographers by letting them tune the CLPD on \textit{in vivo} ultrasound images with the USDG (\textbf{\cref{section:eval}}).
\end{itemize}
All of the code used in our work is openly available online\footnote{\url{https://github.com/Red-Portal/UltrasoundDesignGallery.git}}.

%% The interface of the \usdg~is primarily inspired by 
%% Among many different types of design galleries, \usdg~utilizes the \textit{sequential line search} interface designed by~\cite{10.1145/3072959.3073598}.

%% Design galleries were first developed in the computer graphics community of efficient visual design.

%% While visual tools for understanding the preference of sonographers have been introduced before~\cite{hemmsen_ultrasound_2010}, the \usdg~differs in that it provides 
%% focuses on optimizing and desiging image enhancement algorithms.
%% and  (BO,), a gradient-free global optimization algorithm.

%% reporting the exact image dimensions and view depths is important since they affect the sampling rate the denoising algorithms operate.
%% For example, the window size of the DPAD coefficient.
%% They tend to severly affect the performance of despeckling filters and make proper comparison difficult.

%% Methods such as~\cite{hutchison_probabilisticdriven_2010, bini_despeckling_2014} exploit local homogeneity.
%% However, these methods are difficult to implement in a real-time fashion on highly parallel computing hardware such as GPUs.

%% Speckle reduction algorithms result in improved contrast and higher lesion detectibility~\cite{bottenus_resolution_2021}.
%% The recently introduced generalized contrast-to-noise ratio metric~\cite{rodriguez-molares_generalized_2020}.

%% Laplace regularized PMAD

%and have been integrated into  types of human-computer interfaces~\cite{brochu_bayesian_2010, 10.1145/3072959.3073598, koyama_sequential_2020, phan_color_2018, pmlr-v119-mikkola20a}.

%% For example, in~\cite{ramos-llorden_anisotropic_2015}, Ramos-Llorden et al.~minimized the \(\widehat{Q}\) index for tuning the parameters of various speckle reduction algorithms.

%% While in general optimizing image quality metrics
%% tuning these parameters 
%% However, 
%% While optimization methods are to the rescue 
%% For this reason, 

%% ``\ldots the contrast perceived by the radiologists is far from the one assessed by the three objective metrics used''~\cite{outtas_subjective_2018}.

%% Tools for aiding visual designs have been employed in other fields such as computer graphics~\cite{10.1145/258734.258887}.

%% Despite showing strong speckle reuduction properties, these methods require excessive and tedious tuning~\cite{duarte-salazar_speckle_2020}.
%% For example,~\cite{ramos-llorden_anisotropic_2015} optimized the \(\widehat{Q}\) criterion (originally proposed in~\cite{tay_ultrasound_2006}) for tuning parameters.
%% However, ``OSRAD and POSRAD filters are not able to improve the DPAD filter result mainly due to the higher preservation of structures in the background class'' and ``\ldots in the following experiments with real images, \ldots over-filtering in background images does not necessarily lead to favorable visual results.''.

%% In this paper, we present 

%% \cite{hemmsen_ultrasound_2010} pairwise and continuous comparison however does not perform automatic tuning based on the user feedback.

%% Human-computer interaction for visual design~\cite{tory_human_2004}

%% For high-dimensional spaces,~\cite{10.1145/3386569.3392409} use random linear embeddings~\cite{10.5555/2540128.2540383, NEURIPS2020_10fb6cfa}.


%%% Local Variables:
%%% TeX-master: "master"
%%% End:
