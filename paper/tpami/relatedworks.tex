
\section{Related Works}\label{section:relatedworks}
\subsection{Automatic Tuning of Ultrasound Imaging Systems}
Automatic tuning of the various system parameters in medical ultrasound imaging systems is an important task.
The two fundamental tasks of automatic tuning are to set up an objective function and to choose an optimization procedure.
For example, the weighted sum of edge contrast and soft tissue roughness has been used for tuning the time-gain compensation curve~\cite{lee_automatic_2006} and dynamic range~\cite{lee_automatic_2015}.
For tuning image enhancement algorithms, Tay \textit{et al.} propose the \(\widehat{Q}\)-index metric~\cite{tay_ultrasound_2006} (later used in~\cite{coupe_nonlocal_2009, ramos-llorden_anisotropic_2015}), while Mishra \textit{et al.} used the classic SSNR metric~\cite{mishra_edge_2018}.
Unfortunately, parameters optimal according to these metrics often turn out to be poor on \textit{in vivo} images~\cite{ramos-llorden_anisotropic_2015}.
Therefore, automatic tuning approaches based on objective quality metrics are fundamentally limited.

El-Zehiry \textit{et al.}~\cite{el-zehiry_learning_2013} took a fundamentally different approach.
They first train a machine learning model on the sonographers' subjective quality metrics.
Then, the model guides the tuning process of the system parameters, where a sonographer provides  preferential feedback from a sonographer, where the model guides optimization.
While conceptually similar to ours, the approach of El-Zehiry \textit{et al.} requires a large dataset of subjective assessment.
This dataset needs to be reconstructed from scratch for each ultrasound imaging system, which limits applicability.

For optimization, we leverage Bayesian Optimization~\cite{shahriari_taking_2016} (BO), a black-box gradient-free optimization algorithm.
BO has been successfully applied to tuning systems that are difficult to model mathematically.
Also, BO requires only a small number of objective function evaluations to converge, which sets it apart from metaheuristic approaches.
Therefore, it has been applied to problems where evaluating the objective is expensive, including analog circuit design~\cite{lyu_multiobjective_2018} and drug discovery~\cite{sano_application_2020}.
The use of BO within Design Galleries has been popularized by Brochu \textit{et al.}~\cite{NIPS2007_b6a1085a, brochu_bayesian_2010}.
Connections between Design Galleries and BO can be found in preferential BO~\cite{pmlr-v70-gonzalez17a}, where the objective function is specifically assumed to be human preference.
In this work, we used the projective preferential BO~\cite{pmlr-v119-mikkola20a}, which provides a probabilistic model ideally suited for our design gallery.

\subsection{Subjective Image Quality Assessment}
Subjective quality assessment is an essential tool for evaluating ultrasound imaging techniques.
For example, ultrasound image enhancements algorithms have used subjective quality assessment for accurately evaluating their performance~\cite{loizou_quality_2006, hemmsen_ultrasound_2010, wong_monte_2012, kang_new_2016, mishra_edge_2018}.
However, obtaining reliable subjective quality assessment is challenging both in terms of time and human resources.
Even worse, accurately quantifying subjective quality is a fundamental challenge on its own~\cite{streijl_mean_2016}.
To solve this problem, Hemmsen \textit{et al.} have proposed a software for estimating subjective quality~\cite{hemmsen_ultrasound_2010}.

In the interest of automatic tuning, some works have attempted to learn the subjective quality metrics of sonographers.
As previously discussed, El-Zehiry \textit{et al.} learned the relationship between system parameters and quality assessment and used it for interactive optimization.
More recently, some works trained deep learning models to predict the subjective quality assessment of sonographers~\cite{abdi_automatic_2017, annangi_ai_2020}.
These approaches are limited in that they require a large dataset of assessments.
To train personalized or task-specific models, they need to create a new dataset from scratch.

Our method for inferring subjective quality assessments is based on the Design Gallery~\cite{10.1145/258734.258887}.
So far, we are the first work proposing the use of the Design Gallery 
Design Galleries have been successfully applied to various visual design tasks such as graphics design~\cite{10.1145/258734.258887}, procedural animation generation~\cite{brochu_bayesian_2010}, and color design~\cite{phan_color_2018}.
Alternative user interfaces have been proposed~\cite{phan_color_2018, 10.1145/3072959.3073598, koyama_sequential_2020}; these were mainly designed to be used with still images.
Since medical ultrasound images are typically presented in continuous video sequences, we use the 1D sequential line search interface~\cite{10.1145/3072959.3073598}.


%The works of Brochu \textit{et al.} has later lead to the development of preferential BO~\cite{pmlr-v70-gonzalez17a, pmlr-v119-mikkola20a}, which specifically aim to infer and optimize human preference.
%In this work, we combine projective preferential BO~\cite{pmlr-v119-mikkola20a} with the sequential line search interface, which provide a probabilistic model for Design Galleries with continuously embedded choices.

%% An important issue is that speckle reduction filter tend to generate blurry results.

%% tend to generate blurry images regardless of the speckle reduction performance.
%% Especially, despite the superior performance of the hybrid filter~\cite{singh_hybrid_2017}.

%% Recently, Annangi~\textit{et al.} trained deep learning models to predict the subjective quality assessments of sonographers~\cite{annangi_ai_2020}.
%% The subjective quality metrics were recorded in a 1 to 5 Lickert scale.
%% Deep learning based approach require a large amount of labeled data (their dataset has 320 subjects).
%% Therefore, compared to our approach, task and sonographer specific learning of quality metrics is difficult.


%% Recently,
%% machine learning based a


\subsection{Ultrasound Image Enhancement Algorithms}
Applying image enhancement algorithms is an effective way to improve the perceived quality of medical ultrasound images.
In particular, anisotropic diffusion methods~\cite{perona_scalespace_1990, weickert_anisotropic_1998} have shown excellent performance for ultrasound images~\cite{yongjianyu_speckle_2002, abd-elmoniem_realtime_2002, aja-fernandez_estimation_2006, krissian_oriented_2007, vegas-sanchez-ferrero_probabilisticdriven_2010, ramos-llorden_anisotropic_2015, mishra_edge_2018}.
Combined with the multiscale analysis of Laplacian pyramids~\cite{burt_laplacian_1983}, diffusion methods have demonstrated advanced feature enhancing capabilities~\cite{zhang_multiscale_2006, zhang_nonlinear_2007, kang_new_2016}.
While wavelet decomposition approaches also provide a way to perform multiscale analysis~\cite{xulizong_speckle_1998, xiaohuihao_novel_1999, pizurica_versatile_2003, yongyue_nonlinear_2006}, compared to Laplacian pyramids, they are tricky to combine with other image filters.
Our work demonstrated how to use the Laplacian pyramids in the \textit{cascaded} configuration, which further enhances their flexibility.

Enhancing image structures is a crucial task for ultrasound images.
However, compared to speckle and contrast enhancement, structural enhancement has not received much attention.
Some exceptions include the NCD~\cite{abd-elmoniem_realtime_2002}, which is capable of spatial coherence-enhancement~\cite{weickert_coherenceenhancing_1999}, and shock filters~\cite{zhang_multiscale_2006, kang_new_2016}.
We experimented with different types of shock filters and experienced that the results looked too artificial.
This has also been noted by Kang \textit{et al.}~\cite{kang_new_2016} when discussing the works of Zhang \textit{et al.}~\cite{zhang_multiscale_2006}.
On the other hand, NCD is a powerful tool for enhancing image structures when used in the upper levels of the Laplacian pyramid.

Most anisotropic diffusion filters perform edge detection as an intermediate step.
The low SNR of ultrasound images makes edge detection challenging, harming the performance of conventional diffusion filters.
Although diffusion filters based on Kuan and Lee's coefficient avoid edge-detection~\cite{yongjianyu_speckle_2002, aja-fernandez_estimation_2006, krissian_oriented_2007}, they result in blurry, over-smoothed images~\cite{ramos-llorden_anisotropic_2015, mishra_edge_2018}.
Recently, approaches that avoid edge-detection by probabilistic segmentation~\cite{vegas-sanchez-ferrero_probabilisticdriven_2010, ramos-llorden_anisotropic_2015}, and histogram of oriented gradients in the superpixel domain~\cite{mishra_edge_2018}.
These methods are computationally expensive and challenging to parallelize, impeding their real-time implementation.
However, our Laplacian pyramid-based approach provides a noise-robust way of applying conventional diffusion filters to ultrasound images.
Also, we have demonstrated the utility of RPNCD~\cite{gilboa_image_2004}, which has not been used for ultrasound images, despite being robust against speckle noise.

Meanwhile, some diffusion algorithms have proposed to selectively avoid smoothing pixels belonging to tissues~\cite{ramos-llorden_anisotropic_2015, mishra_edge_2018}.
In our experiments, conventional tissue-selective diffusion turned out to be sensitive to pepper noise.
Also, as discussed in~\cref{section:fourchamber}, tissue selectivity is not always desirable depending on the clinical task.
Therefore, further investigation on tissue selectivity will need to be done.


%% Various approaches for tuning medical ultrasound imaging systems have been proposed.
%% However, most of these approaches were based on optimizing objective quality metrics, which are known to deliver good results.
%% For example, specifically for image enhancement algorithms, Tay \textit{et al.} propose to maximize the \(\widehat{Q}\)-index metric~\cite{tay_ultrasound_2006}.
%% 
%% This significantly restricts its practicality (see~\cite{ramos-llorden_anisotropic_2015}).
%% Unfortuan


%%% Local Variables:
%%% TeX-master: "master"
%%% End:
